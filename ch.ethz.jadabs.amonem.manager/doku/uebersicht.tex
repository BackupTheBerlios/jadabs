\documentclass[10pt]{article}

\usepackage{a4}
\usepackage{color}

\author{Barbara Scheuner, Markus Vitalini, Andreas Scherrer}
\title{jadabs-amonem}

\begin{document}
\maketitle
\tableofcontents

\section{GUI}
\subsection{Oberfl\"ache}
[BILD]

\subsection{Funktionalit\"at}
Das GUI bietet folgende Funktionalit\"at:
\begin{itemize}
	\item Speichern/Laden von Simulationssetups
	\item Anzeigen von (allen) Peers und deren Gruppenstruktur bzw. von Peers pro Gruppe
	\item Erstellen eines neuen zu simulierenden Peers
	\item Rekonfigurieren eines vorhandenen, simulierten Peers
\end{itemize}

\section{Master}
\subsection{Aufgaben}
Der Master ist die ``Anlaufstelle'' f\"ur alle anderen Funktionseinheiten. D.h. der Master weiss
zu jeder Zeit, vieviele Peers vorhanden sind, wie diese heissen, was sie k\"onnen, zu welcher/welchen
Gruppen sie geh\"oren, etc.
\subsection{Schnittstellen}
\subsubsection{Schnittstelle zum GUI}

Interface f\"ur die Darstellung\\\\
\textcolor{red}{Da es sich bei der Gruppenstruktur \emph{nicht} um einen Baum handelt geht das so nicht wirklich (Markus).
Wir brauchen einen DAG.}\\
Markus sieht sich \verb+http://sakharov.net/graph.html+ mal an.
\begin{itemize}
	\item Iterator \"uber alle Peers (gibt ``die'' PeerID und einen Vektor mit den IDs der Gruppen zu
denen er geh\"ort (die ``worldgroup'' ist das erste Element) zur\"uck)\\
Wie soll die Methode heissen?
	\item ``Get\_Info'' f\"ur einen Peer (\"uber die ID), gibt zur\"uck, ob der Peer z.B. eine Bluetooth
Device ist. Weiteres w\"are zu definieren.
	\item ``Get\_Info'' f\"ur eine Gruppe (\"uber die ID), gibt den Namen der Gruppe zur\"uck. Weiteres
w\"are zu definieren.
	\item WIE SIEHT DIE ID AUS (integer, string, \ldots)? Oder sagen wir einfach es ist ein String
(wenn's ein Integer ist machen wir halt einen String draus, Hauptsache es ist eindeutig)?
\end{itemize}

Interface f\"ur neuen Peer\\
noch nicht definiert

\subsubsection{Schnittstelle zum Discoverer}
noch nicht definiert

\subsubsection{Schnittstelle zum Deployer}
noch nicht definiert

\section{Discoverer}
\subsection{Aufgaben}
\subsection{Funktionsweise}

\section{Deployer}
\subsection{Aufgaben}
\subsection{Funktionsweise}

\section{Tipps and Tricks}

\subsection{maven installieren}
\begin{enumerate}
	\item maven herunterladen (von maven.apache.org; Version 1.0, NEU 1.0.1)
	\item osgi-plugin herunterladen \verb+(http://wlab.ethz.ch/maven/repository/maven+\ldots\\
	\verb+/plugins/maven-osgi-plugin-0.3.1.jar)+
	\item maven ``installieren'': unzippen in ein Directory \texttt{(=\$MAVEN\_HOME)}
	\item osgi-plugin kopieren nach \texttt{\$MAVEN\_HOME/plugins}
	\item Umgebungsvariablen und \texttt{\$PATH} setzen (in \verb+~/.bashrc oder ~/.bashrc\_profile?+), in Linux mit \verb+export+:\\
	\texttt{\$MAVEN\_HOME}\\
	\texttt{\$JAVA\_HOME}\\
	\texttt{PATH=\$PATH:$MAVEN\_HOME/bin}
\end{enumerate}

\subsection{Neuen Peer ``installieren''}
Unser Installer muss eine M\"oglichkeit geben, Bundles auszw\"ahlen, diese zuammensuchen (und z.B. zippen\footnote{Wie?}) und dann ``deployen'' (kann remote oder lokal sein, wobei wir das unterscheiden sollten, damit wir lokal nicht alle bundles 100-mal kopieren).\\
Um einen weiteren Peer zu starten generieren wir ein File/eine Zeile wie in runpeer1.sh und ein xargs-File, dann starten wir den Peer mit execute\footnote{das habe ich noch nicht ganz verstanden\ldots V.a. m\"ussen wir im richtigen Verzeichnis sein.}.

\subsection{Interessante Files}
\texttt{jadabs/bundles/maingui/src/MainComposite.java}\\
Hier wird das GUI ``gezeichnet'' und v.a. die Listener installiert (die wiederum Dinge erledigen, die wir auch m\"ussen).

\subsection{n\"utzliche Befehle}
\texttt{java.lang.Runtime.getRuntime()} und/oder \texttt{java.lang.Process proc = rt.exec()}

\subsection{jadabs Co. aus dem CVS holen (Berlioz)}
von \texttt{http://developer.berlios.de/cvs/?group\_id=2534} den Link\\ \texttt{pserver:anonymous@cvs.jadabs.berlios.de:/cvsroot/jadabs} nehmen.
Setzt sich wiefolgt zusammen: \texttt{pserver} bei ``Connection Type'', \texttt{anonymous@cvs.berlios.de} bei host und \texttt{/cvsroot/jadabs} bei Repository Path angeben.

\end{document}
